\chapter{Conclusion}
% Explain what you have learned in this lab. What was difficult, what was surpsrising, and so on.
In Kernel lab, I've learned how to implement kernel module programming.
I've never used kernel information since I am just end user of the system programming.
\texttt{kamlloc} function and \texttt{printk} function are meaningful to me.
I did not understand why I have to use that instead of \texttt{malloc} and \texttt{printf}.
In debugging process(not kernel debugging, my debugging), I use \texttt{dmesg} to show my logs.
\texttt{printk} function was really helpful. 

The most difficult problem was reboot problem.
Unlike the previous project, I have to use Gentoo virtual machine since it is a kernel programming.
After any error, I have to reboot to make the files again.
If not, teminal says there is module running and kill my processor.
Although Virtualbox is quite fast, reboot takes some time and makes me annoying.


The most surpsrising thing was that physical address is too big.
\texttt{sudo ./app} says that my \texttt{pid, vaddr, paddr} value is
\texttt{pid: 3401 vaddr: 12bc010 paddr:8000000070488010}.
I think I might take some mistakes since the most significant bit in \texttt{paddr} is 1, and the following 7 bits are zero.

% Tips
% do not copy-paste screenshots of your code. Format it properly as text
% length: do not exceed 5 pages. There is no lower limit, but one page is probably not detailed enough.