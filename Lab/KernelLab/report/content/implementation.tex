\chapter{Implementation}

% describe your design, implementation, way to solve the lab
% The Kernel lab consists of two phases: ptree and paddr.

The kernel developers have to follow the convention for Linux Kernel Module.
A basic frame for Linux Kernel Module is \texttt{init\char`_module} and \texttt{exit\char`_module}.
The formal is called when the kernel module is inserted to system, and the latter is called when the kernel module is removed from system.
The two functions are enrolled to the kernel using module init and module exit functions.
Here is a basic structure for kernel module programming in \texttt{debugfs}:

\begin{lstlisting}
static ssize_t operation(struct file *fp, 
                         const char __user *user_buffer, 
                         size_t length, 
                         loff_t *position)
{
    // Operation Details
}

static const struct file_operations dbfs_fops = {
    .operation = operation,
};

static int __init dbfs_module_init(void)
{
    // Some Codes
}

static void __exit dbfs_module_exit(void)
{
    // Some Codes
}
\end{lstlisting}

In each assignments, \texttt{ptree} and \texttt{paddr}, it has skeleton C code and build script.
My task is implement to complete each codes.
Fortunately, I did not have to fix the Makefile.

\section{ptree}
% explain what you did to solve this phase in detail
The purpose of this assignment is tracing process from the leaf to \texttt{init} process and logging it using \texttt{debugfs}.
The followings are the steps of tracing process from the leaf to \texttt{init} in \texttt{write\char`_pid\char`_to\char`_pid}.
We initially store current task in the stack, then trace it by using \texttt{curr->real\char`_parent}.
The stored task information is popped into \texttt{stat} buffer.
The printed information in \texttt{stat} is transfered to \texttt{struct debugfs\char`_blob\char`_wrapper *myblob}, which can hold as much data as we want.

\begin{enumerate}
\item Initialize \texttt{buffer}
\begin{lstlisting}
for (i = 0; i < SIZE; i++) {
    stats[i] = '\0';
}
\end{lstlisting}

\item Get \texttt{pid} from \texttt{user\char`_buffer}.
\begin{lstlisting}
pid_t input_pid;
struct pid *pid;
sscanf(user_buffer, "%u", &input_pid);
pid = find_get_pid(input_pid);
\end{lstlisting}


\item Get \texttt{task\char`_struct} from \texttt{pid} of 2.
\begin{lstlisting}
curr = pid_task(pid, PIDTYPE_PID);
\end{lstlisting}

\item To print \texttt{pid} reversely, make a simple stack(or linked list)
\begin{lstlisting}
struct stack {
    struct task_struct *task;
    struct stack *next;
};
struct stack *stack, *temp;
stack = (struct stack *) kmalloc(sizeof(struct stack), GFP_KERNEL);
temp = (struct stack *) kmalloc(sizeof(struct stack), GFP_KERNEL);
temp->next = NULL;
temp->task = curr;
\end{lstlisting}

\item Trace task struct until \texttt{pid} of the task is equal to 1. (In other words, the task is \texttt{init} struct.)
The task information is stored in the stack.
\begin{lstlisting}
while(1) {
    stack->next = temp;
    if (curr->pid == 1) break;
    curr = curr->real_parent;
    stack->task = curr;
    temp = stack;
    stack = (struct stack *) kmalloc(sizeof(struct stack), GFP_KERNEL);
}
stack = stack->next;
\end{lstlisting}

\item After it reaches to \texttt{init}, pop the stack value on the \texttt{stat} buffer.
\begin{lstlisting}
while(1) {
    curr = stack->task;
    length += sprintf(stats + length, "%s (%d)\n", curr->comm, curr->pid);
    stack = stack->next;
    if (stack == NULL) break;
}
\end{lstlisting}

\item Now we complete \texttt{write\char`_pid\char`_to\char`_input}. To use in \texttt{init\char`_module}, we have to save this file operation in \texttt{.write}.
\begin{lstlisting}
static const struct file_operations dbfs_fops = {
    .write = write_pid_to_input,
};
\end{lstlisting}

\item In \texttt{\char`_\char`_init dbfs\char`_module\char`_init} function, define \texttt{debugfs\char`_blob\char`_wrapper} type.
This struct would help us save data what we want, since strings are too big to store.
\begin{lstlisting}
int stats_size;
int struct_size;
dir = debugfs_create_dir("ptree", NULL);
struct_size = sizeof(struct debugfs_blob_wrapper);
stats_size = 8192 * sizeof(char);
\end{lstlisting}

\item The \texttt{debugfs\char`_blob\char`_wrapper} type variable name is \texttt{myblob}.
We initialize it with \texttt{kmalloc} since we cannot we \texttt{malloc}.
Then, store the data and size as we get in the steps 1 to 6.
\begin{lstlisting}
myblob = (struct debugfs_blob_wrapper *) kmalloc(struct_size, GFP_KERNEL);
myblob->data = (void *) stats;
myblob->size = (unsigned long) stats_size;
\end{lstlisting}


\item Now, create the file with permission 644.
It finishes the initializing module.
\begin{lstlisting}
inputdir = debugfs_create_file("input", 0644, dir, NULL, &dbfs_fops);
ptreedir = debugfs_create_blob("ptree", 0644, dir, myblob);
\end{lstlisting}

\item Finally, implement \texttt{exit\char`_module}. Make free \texttt{myblob} to prevent memory leakage and remove created filesa and directory.
\begin{lstlisting}
static void __exit dbfs_module_exit(void) {
    kfree(myblob);
    debugfs_remove(ptreedir);
    debugfs_remove(inputdir);
    debugfs_remove_recursive(dir);
}
\end{lstlisting}
\end{enumerate}

\section{paddr}
% explain what you did to solve this phase in detail
The purpose of this assignment is finding \texttt{pid}, virtual memory address and physical memory address of \texttt{app} file.
Virtual address is a memory address that points the next virtual or physical memory.
Using virtual address, a user can save/load data efficiently like cache.
Thus, the virtual address is not a actual physical memory address and a computer system translates to communicate with virtual and physical memory.
In each process, a process is allocated virtual memory address that has access to the actual physical memory address.
The Gentoo OS used in the Lab has 4-level page table.
By using multi-level page table, the process has more compact page table area in memory than single page table system process.
The task in this assignment is thus in fact get physical address from \texttt{pid} and virtual address. Finding them are exremely easy.
The followings are how I implemented it.

\begin{enumerate}
\item Define data struct to read, then store data from \texttt{user\char`_buffer}.
We can syncronize \texttt{pckt} and \texttt{user\char`_buffer} by pointing the same address.
Now, we get correct \texttt{pid} and \texttt{vaddr}.
\begin{lstlisting}
struct packet{
    pid_t pid;
    unsigned long vaddr;
    unsigned long paddr;
};
pckt = (struct packet *)user_buffer;
\end{lstlisting}

\item Then, get the current task as we implemented in \texttt{ptree.}
\begin{lstlisting}
struct pid *pid;
pid = find_get_pid(pckt->pid);
task = pid_task(pid, PIDTYPE_PID);
\end{lstlisting}

\item Declare four page table offsets, and declare \texttt{page\char`_addr} and \texttt{page\char`_offset}.
\begin{lstlisting}
pgd_t *pgd;
pud_t *pud;
pmd_t *pmd;
pte_t *pte;
unsigned long page_addr = 0;
unsigned long page_offset = 0;
\end{lstlisting}

\item Thanks to \texttt{p*d\char`_offset} API, we can easily get each offsets from the previous offset or \texttt{mm\char`_struct}.
\begin{lstlisting}
pgd = pgd_offset(task->mm, pckt->vaddr);
pud = pud_offset(pgd, pckt->vaddr);
pmd = pmd_offset(pud, pckt->vaddr);
pte = pte_offset_kernel(pmd, pckt->vaddr);
\end{lstlisting}

\item Finally, get physical address.
In this step, we use \texttt{PAGE\char`_MASK} to get \texttt{page\char`_addr} and \texttt{page\char`_offset}.
The physcial address is a combination of page address and page offset.
\begin{lstlisting}
page_addr = pte_val(*pte) & PAGE_MASK;
page_offset = pckt->vaddr & ~PAGE_MASK;
pckt->paddr = page_addr | page_offset;
\end{lstlisting}

\item Create our output file, and don't forget to remove in exit phase.
\begin{lstlisting}
static int __init dbfs_module_init(void) {
    dir = debugfs_create_dir("paddr", NULL);
    output = debugfs_create_file("output", 0644, dir, pckt, &dbfs_fops
    return 0;
}

static void __exit dbfs_module_exit(void) {
		debugfs_remove(output);
		debugfs_remove_recursive(dir);
}

\end{lstlisting}


\end{enumerate}