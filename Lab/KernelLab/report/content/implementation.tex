\chapter{Implementation}

% describe your design, implementation, way to solve the lab
% The Kernel lab consists of two phases: ptree and paddr.

The kernel developers have to follow the convention for Linux Kernel Module.
A basic frame for Linux Kernel Module is \texttt{init\char`_module} and \texttt{exit\char`_module}.
The formal is called when the kernel module is inserted to system, and the latter is called when the kernel module is removed from system.
The two functions are enrolled to the kernel using module init and module exit functions.
Here is a basic structure for kernel module programming in \texttt{debugfs}:

\begin{lstlisting}
static ssize_t operation(struct file *fp, 
                         const char __user *user_buffer, 
                         size_t length, 
                         loff_t *position)
{
    // Operation Details
}

static const struct file_operations dbfs_fops = {
    .operation = operation,
};

static int __init dbfs_module_init(void)
{
    // Some Codes
}

static void __exit dbfs_module_exit(void)
{
    // Some Codes
}
\end{lstlisting}

In each assignments, \texttt{ptree} and \texttt{paddr}, it has skeleton C code and build script.
My task is implement to complete each codes.
Fortunately, I did not have to fix the Makefile.

\section{ptree}
% explain what you did to solve this phase in detail
The purpose of this Assignment is tracing process from the leaf to \texttt{init} process and logging it using \texttt{debugfs}.
ssibal

\begin{lstlisting}
printf("Hello, World!");
\end{lstlisting}

\section{paddr}
% explain what you did to solve this phase in detail