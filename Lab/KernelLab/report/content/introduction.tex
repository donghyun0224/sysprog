\chapter{Kernel Lab}
% Short project description
The goal of the kernel lab is to learn basic kernel module programming and understand the difference between kernel-level programming and user-level programming.
To put the new feature into the kernel, we originally need to recompile the kernel.
In my experience with installing Gentoo, it takes about 2 hours to compile, which is very annoying and risky.
Instead, we can safely add new module with \texttt{debugfs} with very short time, since the modules we added disappear when we reboot the computer.
The goal of this lab is adding two modules \texttt{ptree}, and \texttt{paddr}.
What exactly each module does is introduced in the next chapter.